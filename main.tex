%-------------------------
% Resume in Latex
% Author : Negar Mirgati
% License : MIT
%------------------------

\documentclass[letterpaper,11pt]{article}

\usepackage{latexsym}
\usepackage[empty]{fullpage}
\usepackage{titlesec}
\usepackage{marvosym}
\usepackage[usenames,dvipsnames]{color}
\usepackage{verbatim}
\usepackage{enumitem}
\usepackage[hidelinks]{hyperref}
\usepackage{fancyhdr}
\usepackage[english]{babel}
\usepackage{tabularx}
\usepackage{hyphenat}
\usepackage{fontawesome}
\usepackage[dvipsnames]{xcolor}
\input{glyphtounicode}


%---------- FONT OPTIONS ----------
% sans-serif
% \usepackage[sfdefault]{FiraSans}
% \usepackage[sfdefault]{roboto}
% \usepackage[sfdefault]{noto-sans}
% \usepackage[default]{sourcesanspro}

% serif
% \usepackage{CormorantGaramond}
% \usepackage{charter}


\pagestyle{fancy}
\fancyhf{} % clear all header and footer fields
\fancyfoot{}
\renewcommand{\headrulewidth}{0pt}
\renewcommand{\footrulewidth}{0pt}

% Adjust margins
\addtolength{\oddsidemargin}{-0.5in}
\addtolength{\evensidemargin}{-0.5in}
\addtolength{\textwidth}{1in}
\addtolength{\topmargin}{-.5in}
\addtolength{\textheight}{1.0in}

\urlstyle{same}

\raggedbottom
\raggedright
\setlength{\tabcolsep}{0in}

% Sections formatting
\titleformat{\section}{
  \vspace{-4pt}\scshape\raggedright\large
}{}{0em}{}[\color{black}\titlerule \vspace{-5pt}]

% Ensure that generate pdf is machine readable/ATS parsable
\pdfgentounicode=1

%-------------------------
% Custom commands

\newcommand{\resumeItem}[1]{
  \item\small{
    {#1 \vspace{-2pt}}
  }
}


\newcommand{\resumeSubheading}[4]{
  \vspace{-2pt}\item
    \begin{tabular*}{0.97\textwidth}[t]{l@{\extracolsep{\fill}}r}
      \textbf{#1} & #2 \\
      \textit{\small#3} & \textit{\small #4} \\
    \end{tabular*}\vspace{-7pt}
}


\newcommand{\resumeSubSubheading}[2]{
    \vspace{-2pt}\item
    \begin{tabular*}{0.97\textwidth}{l@{\extracolsep{\fill}}r}
      \textit{\small#1} & \textit{\small #2} \\
    \end{tabular*}\vspace{-7pt}
}


\newcommand{\resumeEducationHeading}[6]{
  \vspace{-2pt}\item
    \begin{tabular*}{0.97\textwidth}[t]{l@{\extracolsep{\fill}}r}
      \textbf{#1} & #2 \\
      \textit{\small#3} & \textit{\small #4} \\
      \textit{\small#5} & \textit{\small #6} \\
    \end{tabular*}\vspace{-5pt}
}


\newcommand{\resumeProjectHeading}[2]{
    \vspace{-2pt}\item
    \begin{tabular*}{0.97\textwidth}{l@{\extracolsep{\fill}}r}
      \small#1 & #2 \\
    \end{tabular*}\vspace{-7pt}
}


\newcommand{\resumeOrganizationHeading}[4]{
  \vspace{-2pt}\item
    \begin{tabular*}{0.97\textwidth}[t]{l@{\extracolsep{\fill}}r}
      \textbf{#1} & \textit{\small #2} \\
      \textit{\small#3}
    \end{tabular*}\vspace{-7pt}
}

\newcommand{\resumeSubItem}[1]{\resumeItem{#1}\vspace{-4pt}}

\renewcommand\labelitemii{$\vcenter{\hbox{\tiny$\bullet$}}$}

\newcommand{\resumeSubHeadingListStart}{\begin{itemize}[leftmargin=0.15in, label={}]}
\newcommand{\resumeSubHeadingListEnd}{\end{itemize}}
\newcommand{\resumeItemListStart}{\begin{itemize}}
\newcommand{\resumeItemListEnd}{\end{itemize}\vspace{-5pt}}

%-------------------------------------------
%%%%%%  RESUME STARTS HERE  %%%%%%%%%%%%%%%%%%%%%%%%%%%%


\begin{document}

%---------- HEADING ----------

\begin{center}
    \textbf{\Huge \scshape Negar Mirgati} \\ \vspace{3pt}
    \small
    \faAt \hspace{.5pt} \href{mailto:mirgati@ualberta.ca}{mirgati@ualberta.ca}
    $|$
    \faLinkedinSquare \hspace{.5pt} \href{https://www.linkedin.com/in/negar-mirgati-4a84a2108/}{LinkedIn}
    $|$
    \faGithub \hspace{.5pt} \href{https://github.com/negarmirgati}{GitHub}
    $|$
    \faStackOverflow \hspace{.5pt} \href{https://stackoverflow.com/users/6287044/winston}{StackOverflow}
    $|$
    \faGlobe \hspace{.5pt} \href{https://negarmirgati.github.io}{Portfolio}
    $|$
    \faMapMarker \hspace{.5pt} \href{https://goo.gl/maps/FE5WmENspj7VT9YU9}{Edmonton, Canada}
\end{center}


%----------- Research Interests -----------

\section{Research Interests}
  \vspace{3pt}
  
$\bullet \: Machine \: Learning \; \; \bullet \: Deep \; Learning \; \; \bullet \: Game \: AI \;\; \bullet \: Procedural \: Content \: Generation \: in \: Games $



%----------- EDUCATION -----------

\section{Education}
  \vspace{3pt}
  \resumeSubHeadingListStart
    
    \resumeEducationHeading
      {University of Alberta
      }{Edmonton, Canada}
      {M.Sc. in Computing Science;   \textbf{GPA: 4.00/4.00}}{Sep 2021 \textbf{--} Sep 2024 (Expected)}


    \resumeSubheading
      {University of Tehran
      }{Tehran, Iran}
      {B.Sc in Computer Hardware Engineering;   \textbf{GPA(last two years): 3.84/4}}{Sep 2015 \textbf{--} Feb 2020}
  \resumeSubHeadingListEnd



%----------- RESEARCH EXPERIENCE -----------

\section{Research Experience}
  \vspace{3pt}
  \resumeSubHeadingListStart
  
    \resumeSubheading
      {Graduate Researcher at \href{https://www.ualberta.ca/computing-science/index.html}{\color{PineGreen} University of Alberta}}{Edmonton, Canada}
      {}{Apr 2022 \textbf{--} Present}
        \resumeItemListStart
            \resumeItem{Working on procedural generation of new levels for platformer-based games using gameplay video. The main purpose of this research is to propose a new method of generating levels for unseen games without annotated datasets - Under the supervision of Prof. Matthew Guzdial}
        \resumeItemListEnd
    
    \resumeSubheading
      {Undergraduate Researcher at \href{https://ece.ut.ac.ir/en} {\color{PineGreen} University of Tehran}}{Tehran, Iran}
      {}{Jun 2019 \textbf{--} Aug 2019}
        \resumeItemListStart
            \resumeItem{Worked on predicting award-winner books on the Goodreads website using various machine learning techniques. For this project, a tabular dataset was created by crawling book information from the Goodreads Website - Under the supervision of Prof. Behnam Bahrak}
        \resumeItemListEnd
    
  \resumeSubHeadingListEnd



%----------- WORK EXPERIENCE -----------

\section{Work Experience}
  \vspace{3pt}
  \resumeSubHeadingListStart
    
    \resumeSubheading
      {Summer Intern at \href{http://www.ipm.ac.ir/}{\color{PineGreen}Institute for Research in Fundamental Sciences (IPM)}}{Tehran Iran}
      {}{Jul 2018 \textbf{--} Oct 2018}
        \resumeItemListStart
            \resumeItem{Worked on implementation of the genetic, memetic, artificial bee colony, simulated annealing, and firefly algorithms for the problem of server placement optimization.}
        \resumeItemListEnd
    

    % \vspace{15pt}
    \resumeSubheading
      {Backend Developer at \href{https://synappsgroup.com/}{\color{PineGreen} SynApps} }{Tehran, Iran}
      {}{Mar 2020 \textbf{--} Oct 2020}
        \resumeItemListStart
            \resumeItem{Worked as a Django Backend Developer on the clinic management application project. The application offers features for organizing crowded patient queues, submitting medical history, lab tests, and radiology reports of the patients.}
        \resumeItemListEnd
    
  \resumeSubHeadingListEnd

%----------- PROJECTS -----------

\section{Notable Projects}
    \vspace{3pt}
    \resumeSubHeadingListStart
        
      \resumeProjectHeading
        {\textbf{Server Placement Optimization} $|$ \emph{\href{https://github.com/NegarMirgati/Sever-Placement-Optimization}{\color{PineGreen}GitHub}}}{}
          \resumeItemListStart
            \resumeItem{Implementation of nature-based algorithms for the problem of server placement optimization (Implemented in Python) - Summer internship project @ IPM}
          \resumeItemListEnd
      
      \resumeProjectHeading
        {\textbf{MAPVis} $|$ \emph{\href{https://github.com/NegarMirgati/MAPVis}{\color{PineGreen}GitHub}}}{}
          \resumeItemListStart
            \resumeItem{A web-based interface used for visualizing the Maximum A-posteriori Estimation (MAP) in data communication (Implemented using Javsascript and HTML/CSS)}
            \resumeItem{The purpose of developing this tool was to help the data communication course students understand the aforementioned concept. (A TAship task for the data communication course @ UT)}
          \resumeItemListEnd

      \resumeProjectHeading
        {\textbf{Light-Seeking Arduino Robot} $|$ \emph{\href{https://github.com/NegarMirgati/Light-Seeking-Robot}{\color{PineGreen}GitHub}}}{}
          \resumeItemListStart
            \resumeItem{Main project of the course Real-Time and Embedded Systems at University of Tehran}
            \resumeItem{Design and development of a robot that simulates a smart moving plant pot, which automatically seeks and moves to lighter places and waters the plant if necessary (Implemented using Arduino .ino (C++)) }
          \resumeItemListEnd
      
    \resumeSubHeadingListEnd

%----------- Teaching Experience -----------
\section{Teaching Experience}
  \vspace{3pt}
  \resumeSubHeadingListStart

    \resumeSubheading
      {CMPUT174 (Lead Student Instructor)
      }{Fall 2022 - University of Alberta}
      {}{}

    \resumeSubheading
      {CMPUT175
      }{Winter 2022 - University of Alberta}
      {}{}

    \resumeSubheading
      {CMPUT174
      }{Fall 2021 - University of Alberta}
      {}{}

    \resumeSubheading
      {Network Security
      }{Spring 2021, Fall 2020 - University of Tehran}
      {}{}

      \resumeSubheading
      {Data Communications
      }{Fall 2020 - University of Tehran}
      {}{}
    
    \resumeSubheading
      {Engineering Probability and Statistics
      }{Fall 2017 - University of Tehran}
      {}{}
  \resumeSubHeadingListEnd

%----------- SKILLS -----------

\section{Technical Skills}
  \vspace{2pt}
  \resumeSubHeadingListStart
    \small{\item{
        \textbf{Programming: }{Python, Java, C++, R} \\ \vspace{3pt}
        
        \textbf{Technologies:}{ Git, Arduino, Quartus, Xilinx ISE, Hspice} \\ \vspace{3pt}       
        \textbf{Frameworks: }{Django, React} \\
        \textbf{ML Frameworks: }{Tensorflow, Keras, Scikit-learn} \\
        % \textbf{Developer Tools}{: X, X, X} \\
        % \textbf{Libraries}{: X, X, X} \\
        % \textbf{Applications}{: X, X, X}
    }}
  \resumeSubHeadingListEnd

  \section{Language Skills}
  \vspace{2pt}
  \resumeSubHeadingListStart
    \small{\item{
        \textbf{English: }{Fluent} 
        \begin{itemize}
            \item TOEFL ibt (Nov 2020) : 110/120
        \end{itemize} \\ \vspace{3pt}
        \textbf{Persian:}{ Native} \\ \vspace{3pt}
        
        \textbf{French: }{Elementary}
    }}
  \resumeSubHeadingListEnd



%----------- RELEVANT COURSEWORK -----------

\section{Relevant Coursework}
  \vspace{2pt}
  \resumeSubHeadingListStart
    \small{\item{
        \textbf{University of Alberta: }{Intro to Machine Learning (A+), Intro to NLP (A), Approximation Algorithms (A), Knowledge Graphs (A+) } \\ \vspace{3pt}
        
        \textbf{University of Tehran (Selective): }{Data Communications (19.5/20), Algorithms Design (19.4/20), Engineering Mathematics (19.5/20), Introduction to Wireless Networks (18.6), Linear Control Systems  (18.5/20), Algorithmic Graph Theory (17.8/20)}


        \textbf{Online Coursework: }{Neural Networks and Deep Leaning (\href{https://coursera.org/share/de361704d1530005cb4d7628d505e023}{\color{PineGreen} DeepLearning.ai}), Introduction to TensorFlow for Artificial Intelligence, Machine Learning, and Deep Learning \href{https://www.coursera.org/account/accomplishments/verify/XJAB2USQJ2EJ?utm_source=link&utm_medium=certificate&utm_content=cert_image&utm_campaign=pdf_header_button&utm_product=course}{\color{PineGreen} (DeepLeaning.ai)}, Convolutional Neural Networks in Tensorflow \href{https://www.coursera.org/account/accomplishments/certificate/3WY464Q3JZTG}{\color{PineGreen} (DeepLearning.ai) }
    }}
  \resumeSubHeadingListEnd


%-------------------------------------------
\end{document}
